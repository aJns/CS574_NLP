\section{Team details}
\begin{description}
  \item[Team:] Optimistic Cat Monkeys
  \item[Division of labor:] At the moment we are planning to split the work
    quite equally; everybody will take part in planning, implementation and
    documentation.
  \item[Diversity:] 3 different nationalities. One female member.
\end{description}


\section{Problem setup and motivation}

The problem we want to solve is summarising a long text, such as a novel or screenplay. We formally define this as summarising a monolingual (English) single-document text of over 50,000 words to one under 1000 words, while retaining the important essence of the information.\\
We think this is an interesting problem to solve, because summarising short articles is a fairly common application of NLP, but algorithms that can summarize long pieces are less present in literature. Even summarising multiple short documents, such as emails and news articles, is also an application receiving attention. Columbia Newsblaster is one popular news summary available online. \\
The applications of automatic book summaries specifically include it being less biased than people, and it can help decide whether to read/purchase a book through a quick overview. Additionally, it could help refresh one's memory of a book they read before, or help determine whether a certain reference book is relevant.\\
Thus, we have a place to start (short article summary), but still have to come up with new solutions. There exists a sizeable body of literature focusing on automatic text summarization, and a useful starting point is recent review papers. Upon gaining some more experience in NLP methods, we can then compare the various approaches in the review papers, both with machine learning and without, and choose what seems like a feasible first method. \cite{Kumar2016}


\section{Proposed approach}

Existing automatic text summary algorithms fall into two general approaches: extractive and abstractive. Extractive methods involve identifying the most important phrases and compiling them verbatim into a shorter text. This method is the simpler, but reads less naturally and might perform even less effectively when scaled to large text. Abstractive methods identify the most important information, and rephrase them into a coherent summary. This method is more challenging because it involves identifying information as well as generating text. However, it reduces redundancy and it reads more naturally, also being similar to how humans summarise texts. \cite{Gaikwad2016} \\
The project approach would probably work best by building in complexity, creating working algorithms and then progressing to the next goal: from extractive to abstractive, and from short text to long ones. The summarization would be based on features such as term frequency, similarity and proper nouns. With article summarization, there is a heavy emphasis on comparing text to the title, which works well because the title normally is a very concise wording of the most important idea. However, with books this is usually not the case, so we must also develop better ways to identify information as important. PoS tagging might be useful in this, as proper nouns and verbs tend to carry a lot of information. Another possibility would be to summarise all the shorter texts, for example reduce each page to a sentence, and then repeat this recursively until the desired length is reached.

\section{Evaluation plan}
A summary could be judged on two main aspects: its retention of important content and the readability. However, quantitative measures are not ideal because there are many possible good answers, and "good" in this case is also subjective. Nevertheless, two methods are used in practice to evaluate summaries, and will be used in this project as well: ROUGE and BLEU metrics. ROUGE and BLEU are measures that compute variations of precision and recall of N-gram overlaps between the summary and the references \cite{nenkova2006summarization} \cite{lin2004rouge}. \\
Boths these metrics require a golden standard in the form of human-written summaries, and these will be taken from free online resources such as WikiSummaries, SparkNotes and CliffsNotes. A larger reference set is better, but realistically 2 summaries are expected to be found for each book on average, with the goal of at evaluating at least 10 book summaries. \\
These cover content, but for small reference sets, would not be able to cover grammar and readability. A small number of these could be read by the project members themselves and evaluated quantitatively for cohesion and readability, although this is not expected to be of a high quality based on knowledge of current NLP performance, because this is a challenge even in state-of-the-art systems.

\section{Plan of work}

\begin{enumerate}
  \item Get a simple summarising system that can summarize short texts working, by taking some currently used techniques
  \item Get another - or come up with another - system that can summarize long text to relatively shorter text with preserving the semantics.
  \item Connect those two systems (or multiple of systems in layer) to summarize long text to be much shorter.
  \item Adopt Evaluation plan used in semantic based approaches in abstractive methods to gauge accuracy of our system. (Up to here is planned to be done until next proposal)
  \item If it turned out to be no way to evaluate our system, then consider pivoting of the idea from the beginning
  \item If it seems producing some meaningful results, then stick to the initial plan
\end{enumerate}

