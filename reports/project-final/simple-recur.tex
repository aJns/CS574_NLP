\subsection{Simple, recursive summarization}

We thought that a simple way to do extractive summarization on a large body of
text would be to employ a basic summarizer, but to apply it
\textit{recursively}. Our recursive summarizer ``meta-script'' takes a
summarizing function, which can basically be anything, and applies it
recursively. First it divides the text into chunks of \(n\) sentences and runs
the summarizer on those chunks. The resulting summaries are then combined, and
fed as the new input to the recursive summarizer. At the start of every
recursion we check the length of the text, stopping if it's short enough.

With a little glue code, our recursive python script can be made to use any
summarizing function. In our experiments, we used a very simple summarizing
function; It counted the word frequencies, and using those frequencies it
assigned a score to each sentence, selecting those that had the highest scores.

In previous stages of the project we looked into the different nouns appearing
in texts. We compiled some noun frequency data, and found out that
often the most used noun referred to the main character of a book (see
figures in~\ref{appendix:countfigs}), which isn't
admittedly that surprising. In the first version of our recursive algorithm the
summarizer only looks at the words in the current block; We thought it would be
interesting to try to add the frequencies of the most used nouns. 

In the final version the code first compiles a dictionary of nouns and their
frequencies. That noun frequency data can be added to the word frequency data
obtained from each block. The importance of the noun frequencies can also be
adjusted.
